\documentclass{article}
\usepackage[utf8]{inputenc}

\title{Title}
\author{André Martiny}
\date{}
\usepackage{lineno}
\linenumbers
\usepackage{natbib}
\usepackage{graphicx}

\usepackage[utf8]{inputenc}
%\usepackage[norsk]{babel}                  % Oversetter overskrifter/abstract osv til norsk.
\usepackage[T1]{fontenc}
\usepackage[bitstream-charter]{mathdesign}	% Velger Charter-fonten for både tekst og matematikk.
\usepackage{centernot}                      % Gir mulighet for ikke-implikasjoner
\usepackage[toc,page]{appendix}             % Gir mulighet for Appendix
\usepackage{graphicx}
\usepackage{scalerel}

\usepackage{hyperref}
\hypersetup{
	colorlinks=true,
	pdftitle={Oppslagstavle},
    pdfauthor={AA}
}

\usepackage{enumitem}						% Gjør at vi kan finjustere lister
\setlist{nosep}								% Fjerner overflødig mellomrom mellom liste-items.
\usepackage{amsmath}
% \usepackage{amssymb}
\usepackage{amsthm}

\usepackage[dvipsnames, svgnames]{xcolor}	% Gjør at vi kan bruke farger
\usepackage{tcolorbox}						% Gjør at vi kan bruke fancy bokser
\tcbuselibrary{breakable}					% Gjør at boksene kan kan dekke flere sider
\usepackage[useregional]{datetime2}			% Gjør at vi kan standardisere datoformatet
\usepackage{colonequals}
% Her definerer vi masse kommandoer:
\DeclareMathOperator{\sign}{sign}
\newcommand{\msn}{\mathbb{N}}
\newcommand{\msz}{\mathbb{Z}}
\newcommand{\msq}{\mathbb{Q}}
\newcommand{\msr}{\mathbb{R}}
\newcommand{\msc}{\mathbb{C}}
\newcommand{\N}{\mathbb{N}}
\newcommand{\Z}{\mathbb{Z}}
\newcommand{\Q}{\mathbb{Q}}
\newcommand{\C}{\mathbb{C}}
\newcommand{\R}{\mathbb{R}}
\newcommand{\dual}{^*}
\newcommand{\notimplies}{\centernot\implies}
\renewcommand{\epsilon}{\varepsilon}
\newcommand*{\defeq}{\mathrel{\vcenter{\baselineskip0.5ex \lineskiplimit0pt
                     \hbox{\scriptsize.}\hbox{\scriptsize.}}}%
                     =} % Denne gir muligheten til å skrive := pent
\let\savestar\star
\renewcommand\star{{\scalerel*{\bigstar}{\savestar}}}


% Her definerer vi masse omgivelser
\newtheorem{thm}{Theorem}%[section]      % Inkluder section hvis du har sections
\newtheorem{prop}[thm]{Proposition}
\newtheorem{lem}[thm]{Lemma}
\newtheorem{cor}[thm]{Corollary}

\theoremstyle{definition}
\newtheorem{defn}[thm]{Definition}
\newtheorem{notation}[thm]{Notation}
\newtheorem{example}[thm]{Example}
\newtheorem{conj}[thm]{Conjecture}
\newtheorem{prob}[thm]{Problem}
\newtheorem{quest}{Question}

\theoremstyle{remark}
\newtheorem{fact}[thm]{Fact}
\newtheorem{claim}[thm]{Claim}
\newtheorem{rem}[thm]{Remark}

% Her definerer vi bluebox-miljøet.
\newenvironment{bluebox}[1]
{
	\begin{tcolorbox}
    [
%     	colback=GhostWhite
    	colback=SkyBlue!10!White,
        colbacktitle=SkyBlue,
        colframe=Black,
        coltitle=Black,
        fonttitle=\bfseries,
        top=0pt,
        boxrule=1pt,
        sharp corners,
        title=#1,
        subtitle style={},
        breakable
    ]
    \setlength{\parskip}{\baselineskip}
}{
	\end{tcolorbox}
}

\newenvironment{yellowbox}[1]
{
	\begin{tcolorbox}
    [
    	colback=Yellow!10!White,
        colbacktitle=Yellow,
        colframe=Black,
        coltitle=Black,
        fonttitle=\bfseries,
        top=0pt,
        boxrule=1pt,
        sharp corners,
        title=#1,
        subtitle style={},
        breakable
    ]
    \setlength{\parskip}{\baselineskip}
}{
	\end{tcolorbox}
}

\newenvironment{greenbox}[1]
{
	\begin{tcolorbox}
    [
    	colback=LimeGreen!10!White,
        colbacktitle=LimeGreen,
        colframe=Black,
        coltitle=Black,
        fonttitle=\bfseries,
        top=0pt,
        boxrule=1pt,
        sharp corners,
        title=#1,
        subtitle style={},
        breakable
    ]
    \setlength{\parskip}{\baselineskip}
}{
	\end{tcolorbox}
}

% Her definerer vi noskip-miljøet:
\newenvironment{noskip}
{
	\setlength{\parskip}{0pt}
}{}

% Disse kommandoene bestemmer lengdene som blir brukt til avsnitt
\setlength{\parskip}{\baselineskip}
\setlength{\parindent}{0pt}
\begin{document}

\maketitle
\begin{defn}
Let $X$ be a Banach space. For $x\in S_X$ and $\epsilon>0$ we define $\Delta_\epsilon(x) = \{y\in B_X \,:\, \|x-y\| \geq 2-\epsilon\}$.
\begin{enumerate}[label={(\roman*)}]
    \item $x$ is a $\Delta$-point if $x\in \overline{conv}{\Delta_\epsilon(x)}$ for all $\epsilon>0$. \\
    \item $x$ is Daugavet-point if $B_X =  \overline{conv}{\Delta_\epsilon(x)}$ for all $\epsilon>0$. 
\end{enumerate}
\end{defn}
Equivalently we can define $\Delta$-points and Daugavet-points in tihs manner
\begin{defn}
Let $X$ be a Banach space. For $x\in S_X$ and $\epsilon>0$ we define $\Delta_\epsilon(x) = \{y\in B_X \,:\, \|x-y\| \geq 2-\epsilon\}$.
\begin{enumerate}[label={(\roman*)}]
    \item $x$ is a $\Delta$-point if for all $\epsilon>0$ and for all $x^*\in S_{X^*}$ with $x\in S(x^*, \delta)$ there exists $y\in S(x^*, \delta)$ with $\|x-y\|\geq 2-\epsilon$\\
    \item $x$ is Daugavet-point if for all $\epsilon>0$, for all $z\in S_X$ and for all $x^*\in S_{X^*}$ with $x\in S(x^*, \delta)$ there exists $y\in S(x^*, \delta)$ with $\|z-y\|\geq 2-\epsilon$
\end{enumerate}
\end{defn}
\begin{defn}

$\Delta_X = \{x \in S_X \,:\, x \text{ is a } \Delta-point \}$.

If $\Delta_X = S_X$ : $X$ has diametral LD2P.

$D_X = \{x \in D_C \,:\, x \text{ is a } D-point\}$.

If $D_x = S_X$ : $X$ has the Daugavet property.
\end{defn}

\textbf{Question:}

What spaces satisfies $\Delta = \emptyset$? Polyhedral? M-embedded? Lorentz?
\textbf{EX:}

$c_0, l_p$ and many polyhedral spaces, finite dimensional spaces, uniformly non-square spaces, 



%%%             C[0,1]              %%%

\begin{lem}
$\Delta = S_{C[0,1]}$.
\end{lem}
\begin{proof}
Idea:  For any $x\in S_{C[0,1]}$ and any $\epsilon >0$. If we take an arbitrary slice $S( x^*, \delta)$ containing $x$, then we can find an interval $(t_0-\epsilon_0, t_0\epsilon)$ ($t_0$ is an endpoint choose an intervall containing $t_0$ to the right or left) with $x(t_0) = 1$ (or -1) and $x(t)\geq 1-\delta$ for all $t\in (t_0-\epsilon_0, t_0\epsilon)$, we now only need to find some closed interval $A$ contained in $(t_0-\epsilon_0, t_0\epsilon)$ with arbitrary small measure (the measure related to $x^*$). Then we let $y$ be the modified version of $x$ such that at the left endpoint of $A$, $x$ is replaced by the line down to $-1$ and the line up to the endpoint of $A$ again.  
\end{proof}

%%%               C_0               %%%

\begin{lem}
$\Delta_{c_0} = \emptyset$.
\end{lem}

\begin{proof}
Take any $x\in S_{c_0}$. We need only find some slice where $\Delta$-point property is not satisfied. Let $\epsilon_0>0$ and let $F_{\epsilon_0} \colonequals \{n\in \N \,:\, |x_n|> 1-\epsilon\}$ and let $M_0$ be the cardinality of $F_{\epsilon_0}$. Notice that if $0<\epsilon<\epsilon_0$ is chosen. Then the cardinality $M$ of $F_\epsilon$ does not increase. We can therefore choose $\epsilon>0$ such that 
\[ 
\epsilon < \frac{1-\epsilon}{M} \implies 1-\epsilon > 1- \frac{1-\epsilon}{M}.
\]
Define $(a_k)$ such that $a_k = \frac{\sign x_k}{M}$ if $k\in F_\epsilon$ and $a_k=0$ otherwise. Then $a \in S_{c_o^*}$ with $a(x)> 1-\epsilon$ and $a(x)- \frac{|x_k|}{M} < 1 - \frac{1-\epsilon}{M}$, for $k\in F_\epsilon$. This means that $x\in S(a, \frac{1-\epsilon}{M})$, but there does not exist $y\in S(a, \frac{1-\epsilon}{M})$ with $\|x-y\|>2-\epsilon$ as this would imply that $\sign y_k = - \sign x_k$ with $|y_k| > 1-\epsilon$ for some $k\in F_\epsilon$. 
\end{proof}



%%%                 c               %%%%


\begin{lem}
$x\in S_c$ with $lim x_n = \pm 1$ implies that $x\in \Delta_c$. $x\in \Delta_c \iff \lim x_n \rightarrow \pm 1$. 
\end{lem}


\begin{proof}
The techniques from the previous lemma show that if $\lim x_n \neq \pm 1$ for some $x$ then $x\notin \Delta$. The proof that $x$ with $lim x_n \rightarrow \pm 1$ is similar to the proof that $c_0$ has LD2P.
\end{proof}

\bibliographystyle{plain}
\bibliography{references}

\end{document}
