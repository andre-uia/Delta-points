\usepackage[utf8]{inputenc}
%\usepackage[norsk]{babel}                  % Oversetter overskrifter/abstract osv til norsk.
\usepackage[T1]{fontenc}
\usepackage[bitstream-charter]{mathdesign}	% Velger Charter-fonten for både tekst og matematikk.
\usepackage{centernot}                      % Gir mulighet for ikke-implikasjoner
\usepackage[toc,page]{appendix}             % Gir mulighet for Appendix
\usepackage{graphicx}
\usepackage{scalerel}

\usepackage{hyperref}
\hypersetup{
	colorlinks=true,
	pdftitle={Oppslagstavle},
    pdfauthor={AA}
}

\usepackage{enumitem}						% Gjør at vi kan finjustere lister
\setlist{nosep}								% Fjerner overflødig mellomrom mellom liste-items.
\usepackage{amsmath}
% \usepackage{amssymb}
\usepackage{amsthm}

\usepackage[dvipsnames, svgnames]{xcolor}	% Gjør at vi kan bruke farger
\usepackage{tcolorbox}						% Gjør at vi kan bruke fancy bokser
\tcbuselibrary{breakable}					% Gjør at boksene kan kan dekke flere sider
\usepackage[useregional]{datetime2}			% Gjør at vi kan standardisere datoformatet
\usepackage{colonequals}
% Her definerer vi masse kommandoer:
\DeclareMathOperator{\sign}{sign}
\newcommand{\msn}{\mathbb{N}}
\newcommand{\msz}{\mathbb{Z}}
\newcommand{\msq}{\mathbb{Q}}
\newcommand{\msr}{\mathbb{R}}
\newcommand{\msc}{\mathbb{C}}
\newcommand{\N}{\mathbb{N}}
\newcommand{\Z}{\mathbb{Z}}
\newcommand{\Q}{\mathbb{Q}}
\newcommand{\C}{\mathbb{C}}
\newcommand{\R}{\mathbb{R}}
\newcommand{\dual}{^*}
\newcommand{\notimplies}{\centernot\implies}
\renewcommand{\epsilon}{\varepsilon}
\newcommand*{\defeq}{\mathrel{\vcenter{\baselineskip0.5ex \lineskiplimit0pt
                     \hbox{\scriptsize.}\hbox{\scriptsize.}}}%
                     =} % Denne gir muligheten til å skrive := pent
\let\savestar\star
\renewcommand\star{{\scalerel*{\bigstar}{\savestar}}}


% Her definerer vi masse omgivelser
\newtheorem{thm}{Theorem}%[section]      % Inkluder section hvis du har sections
\newtheorem{prop}[thm]{Proposition}
\newtheorem{lem}[thm]{Lemma}
\newtheorem{cor}[thm]{Corollary}

\theoremstyle{definition}
\newtheorem{defn}[thm]{Definition}
\newtheorem{notation}[thm]{Notation}
\newtheorem{example}[thm]{Example}
\newtheorem{conj}[thm]{Conjecture}
\newtheorem{prob}[thm]{Problem}
\newtheorem{quest}{Question}

\theoremstyle{remark}
\newtheorem{fact}[thm]{Fact}
\newtheorem{claim}[thm]{Claim}
\newtheorem{rem}[thm]{Remark}

% Her definerer vi bluebox-miljøet.
\newenvironment{bluebox}[1]
{
	\begin{tcolorbox}
    [
%     	colback=GhostWhite
    	colback=SkyBlue!10!White,
        colbacktitle=SkyBlue,
        colframe=Black,
        coltitle=Black,
        fonttitle=\bfseries,
        top=0pt,
        boxrule=1pt,
        sharp corners,
        title=#1,
        subtitle style={},
        breakable
    ]
    \setlength{\parskip}{\baselineskip}
}{
	\end{tcolorbox}
}

\newenvironment{yellowbox}[1]
{
	\begin{tcolorbox}
    [
    	colback=Yellow!10!White,
        colbacktitle=Yellow,
        colframe=Black,
        coltitle=Black,
        fonttitle=\bfseries,
        top=0pt,
        boxrule=1pt,
        sharp corners,
        title=#1,
        subtitle style={},
        breakable
    ]
    \setlength{\parskip}{\baselineskip}
}{
	\end{tcolorbox}
}

\newenvironment{greenbox}[1]
{
	\begin{tcolorbox}
    [
    	colback=LimeGreen!10!White,
        colbacktitle=LimeGreen,
        colframe=Black,
        coltitle=Black,
        fonttitle=\bfseries,
        top=0pt,
        boxrule=1pt,
        sharp corners,
        title=#1,
        subtitle style={},
        breakable
    ]
    \setlength{\parskip}{\baselineskip}
}{
	\end{tcolorbox}
}

% Her definerer vi noskip-miljøet:
\newenvironment{noskip}
{
	\setlength{\parskip}{0pt}
}{}

% Disse kommandoene bestemmer lengdene som blir brukt til avsnitt
\setlength{\parskip}{\baselineskip}
\setlength{\parindent}{0pt}